% !TEX root = ./main.tex

\sjtusetup{
  %
  %******************************
  % 注意:
  %   1. 配置里面不要出现空行
  %   2. 不需要的配置信息可以删除
  %******************************
  %
  % 信息录入
  %
  info = {%
    %
    % 标题
    %
    zh / title           = {基于静态单赋值中间语言的函数式编译器验证方法},
    en / title           = {Verification of Functional Compilers Based on Static Single Assignment Intermediate Representation},
    %
    % 标题页标题
    %   可使用“\\”命令手动控制换行
    % 
    % zh / display-title   = {上海交通大学学位论文\\ \LaTeX{} 模板示例文档},
    % en / display-title   = {A Sample Document \\ for \LaTeX-based SJTU Thesis Template},
    %
    % 关键词
    %
    zh / keywords        = {编译器验证, 形式化方法, 静态单赋值, 函数式编译器},
    en / keywords        = {Compiler Verification, Formal Methods, Static Single Assignment, Functional Compilers},
    %
    % 姓名
    %
    zh / author          = {刘思雨},
    en / author          = {Siyu Liu},
    %
    % 指导教师
    %
    zh / supervisor      = {汪宇霆副教授},
    en / supervisor      = {Assoc. Prof. Yuting Wang},
    %
    % 副指导教师
    %
    % assoc-supervisor  = {某某教授},
    % assoc-supervisor* = {Prof. Uom Uom},
    %
    % 学号
    %
    id              = {121033910117},
    %
    % 学位
    %   本科生不需要填写
    %
    zh / degree          = {工学硕士},
    en / degree          = {Master of Engineering},
    %
    % 专业
    %
    zh / major           = {计算机技术},
    en / major           = {Computer Technology},
    %
    % 所属院系
    %
    zh / department      = {计算机科学与工程系},
    en / department      = {Depart of Computer Science and Technology},
    %
    % 答辩日期
    %   使用 ISO 格式 (yyyy-mm-dd);默认为当前时间
    %
    % date                 = {2023-05-18},
    %
    % 标题页显示日期
    %   覆盖对应标题页的日期显示,原样输出
    %
    % zh / display-date    = {2023 年 5 月},
    %
    % 资助基金
    %
    % zh / fund  = {
    %                {国家 973 项目 (No. 2025CB000000)},
    %                {国家自然科学基金 (No. 81120250000)},
    %              },
    % en / fund  = {
    %                {National Basic Research Program of China (Grant No. 2025CB000000)},
    %                {National Natural Science Foundation of China (Grant No. 81120250000)},
    %              },
  },
  %
  % 风格设置
  %
  style = {%
    %
    % 论文标题页 logo 颜色 (red/blue/black)
    %
    % title-logo-color = black,
  },
  %
  % 名称设置
  %
  name = {
    % bib             = {References},
    % ack             = {谢\hspace{\ccwd}辞},
    % achv            = {攻读学位期间完成的论文},
  },
}

% 使用 BibLaTeX 处理参考文献
%   biblatex-gb7714-2015 常用选项
%     gbnamefmt=lowercase     姓名大小写由输入信息确定
%     gbpub=false             禁用出版信息缺失处理
\usepackage[backend=biber,style=gb7714-2015]{biblatex}
% 文献表字体
% \renewcommand{\bibfont}{\zihao{5}\fixedlineskip{15.6bp}}
% 文献表条目间的间距
\setlength{\bibitemsep}{0pt}
% 导入参考文献数据库
\addbibresource{refs.bib}

% 脚注格式
\usepackage[perpage,bottom,hang]{footmisc}

% 定义图片文件目录与扩展名
\graphicspath{{figures/}}
\DeclareGraphicsExtensions{.pdf,.eps,.png,.jpg,.jpeg}

% 确定浮动对象的位置,可以使用 [H],强制将浮动对象放到这里(可能效果很差)
% \usepackage{float}

% 固定宽度的表格
% \usepackage{tabularx}

% 使用三线表:toprule,midrule,bottomrule。
\usepackage{booktabs}

% 表格中支持跨行
\usepackage{multirow}

% 表格中数字按小数点对齐
\usepackage{dcolumn}
\newcolumntype{d}[1]{D{.}{.}{#1}}

% 使用长表格
\usepackage{longtable}

% 附带脚注的表格
\usepackage{threeparttable}

% 附带脚注的长表格
\usepackage{threeparttablex}

% 算法环境宏包
\usepackage[ruled,vlined,linesnumbered]{algorithm2e}
% \usepackage{algorithm, algorithmicx, algpseudocode}

% 代码环境宏包
\usepackage{listings}
\usepackage{xcolor}
\definecolor{dkgreen}{rgb}{0,0.6,0}
\definecolor{ltblue}{rgb}{0,0.4,0.4}
\definecolor{dkviolet}{rgb}{0.3,0,0.5}
% lstlisting coq style (inspired from a file of Assia Mahboubi)
\lstdefinelanguage{Coq}{ 
    % Anything betweeen $ becomes LaTeX math mode
    mathescape=true,
    % Comments may or not include Latex commands
    texcl=false, 
    % Vernacular commands
    morekeywords=[1]{Section, Module, End, Require, Import, Export,
        Variable, Variables, Parameter, Parameters, Axiom, Hypothesis,
        Hypotheses, Notation, Local, Tactic, Reserved, Scope, Open, Close,
        Bind, Delimit, Definition, Let, Ltac, Fixpoint, CoFixpoint, Add,
        Morphism, Relation, Implicit, Arguments, Unset, Contextual,
        Strict, Prenex, Implicits, Inductive, CoInductive, Record,
        Structure, Canonical, Coercion, Context, Class, Global, Instance,
        Program, Infix, Theorem, Lemma, Corollary, Proposition, Fact,
        Remark, Example, Proof, Goal, Save, Qed, Defined, Hint, Resolve,
        Rewrite, View, Search, Show, Print, Printing, All, Eval, Check,
        Projections, inside, outside, Def},
    % Gallina
    morekeywords=[2]{forall, exists, exists2, fun, fix, cofix, struct,
        match, with, end, as, in, return, let, if, is, then, else, for, of,
        nosimpl, when},
    % Sorts
    morekeywords=[3]{Type, Prop, Set, true, false, option},
    % Various tactics, some are std Coq subsumed by ssr, for the manual purpose
    morekeywords=[4]{pose, set, move, case, elim, apply, clear, hnf,
        intro, intros, generalize, rename, pattern, after, destruct,
        induction, using, refine, inversion, injection, rewrite, congr,
        unlock, compute, ring, field, fourier, replace, fold, unfold,
        change, cutrewrite, simpl, have, suff, wlog, suffices, without,
        loss, nat_norm, assert, cut, trivial, revert, bool_congr, nat_congr,
        symmetry, transitivity, auto, split, left, right, autorewrite},
    % Terminators
    morekeywords=[5]{by, done, exact, reflexivity, tauto, romega, omega,
        assumption, solve, contradiction, discriminate},
    % Control
    morekeywords=[6]{do, last, first, try, idtac, repeat},
    % Comments delimiters, we do turn this off for the manual
    morecomment=[s]{(*}{*)},
    % Spaces are not displayed as a special character
    showstringspaces=false,
    % String delimiters
    morestring=[b]",
    morestring=[d]’,
    % Size of tabulations
    tabsize=3,
    % Enables ASCII chars 128 to 255
    extendedchars=false,
    % Case sensitivity
    sensitive=true,
    % Automatic breaking of long lines
    xleftmargin       = 1em,
    xrightmargin      = 1em,
    breaklines=false,
    framexleftmargin  = 1em,
    framexrightmargin = 1em,
    keepspaces        = true,
    backgroundcolor   = \color{gray!10},
    % Default style fors listings
    basicstyle=\normalsize,
    % Position of captions is bottom
    captionpos=b,
    % flexible columns
    columns=[l]flexible,
    % Style for (listings') identifiers
    identifierstyle={\ttfamily\color{black}},
    % Style for declaration keywords
    keywordstyle=[1]{\ttfamily\color{dkviolet}},
    % Style for gallina keywords
    keywordstyle=[2]{\ttfamily\color{dkgreen}},
    % Style for sorts keywords
    keywordstyle=[3]{\ttfamily\color{ltblue}},
    % Style for tactics keywords
    keywordstyle=[4]{\ttfamily\color{ltblue}},
    % Style for terminators keywords
    keywordstyle=[5]{\ttfamily\color{dkred}},
    %Style for iterators
    %keywordstyle=[6]{\ttfamily\color{dkpink}},
    % Style for strings
    stringstyle=\ttfamily,
    % Style for comments
    commentstyle={\ttfamily\color{dkgreen}},
    %moredelim=**[is][\ttfamily\color{red}]{/&}{&/},
    literate=
    {\\forall}{{\color{dkgreen}{$\forall\;$}}}1
    {\\exists}{{$\exists\;$}}1
    {<-}{{$\leftarrow\;$}}1
    {=>}{{$\Rightarrow\;$}}1
    {==}{{\code{==}\;}}1
    {==>}{{\code{==>}\;}}1
    %    {:>}{{\code{:>}\;}}1
    {->}{{$\rightarrow\;$}}1
    {<->}{{$\leftrightarrow\;$}}1
    {<==}{{$\leq\;$}}1
    {\#}{{$^\star$}}1 
    {\\o}{{$\circ\;$}}1 
    {\@}{{$\cdot$}}1 
    {\/\\}{{$\wedge\;$}}1
    {\\\/}{{$\vee\;$}}1
    {++}{{\code{++}}}1
    {~}{{$\sim$}}1
    {\@\@}{{$@$}}1
    {\\mapsto}{{$\mapsto\;$}}1
    {\\hline}{{\rule{\linewidth}{0.5pt}}}1
    %
}[keywords,comments,strings]
\lstdefinestyle{lstStyleCode}{%
  aboveskip         = \medskipamount,
  belowskip         = \medskipamount,
  basicstyle        = \ttfamily\zihao{6},
  commentstyle      = \slshape\color{black!60},
  stringstyle       = \color{green!40!black!100},
  keywordstyle      = \bfseries\color{blue!50!black},
  extendedchars     = false,
  upquote           = true,
  tabsize           = 2,
  showstringspaces  = false,
  xleftmargin       = 1em,
  xrightmargin      = 1em,
  breaklines        = false,
  framexleftmargin  = 1em,
  framexrightmargin = 1em,
  backgroundcolor   = \color{gray!10},
  columns           = flexible,
  keepspaces        = true,
  texcl             = true,
  mathescape        = true
}
\lstnewenvironment{codeblock}[1][]{%
  \lstset{style=lstStyleCode,#1}}{}

% 直立体数学符号
\providecommand{\dd}{\mathop{}\!\mathrm{d}}
\providecommand{\ee}{\mathrm{e}}
\providecommand{\ii}{\mathrm{i}}
\providecommand{\jj}{\mathrm{j}}

% 国际单位制宏包
\usepackage{siunitx}

% 定理环境宏包
\usepackage{ntheorem}
% \usepackage{amsthm}

% 绘图宏包
\usepackage{tikz}
\usetikzlibrary{arrows.meta, shapes.geometric, chains, shadows.blur}

% 一些文档中用到的 logo
\usepackage{hologo}
\providecommand{\XeTeX}{\hologo{XeTeX}}
\providecommand{\BibLaTeX}{\textsc{Bib}\LaTeX}

% 借用 ltxdoc 里面的几个命令方便写文档
\DeclareRobustCommand\cs[1]{\texttt{\char`\\#1}}
\providecommand\pkg[1]{{\sffamily#1}}

% hyperref 宏包在最后调用
\usepackage{hyperref}

% E-mail
\providecommand{\email}[1]{\href{mailto:#1}{\urlstyle{tt}\nolinkurl{#1}}}

% \usepackage{amssymb}
\usepackage{graphicx}
\usepackage[misc,geometry]{ifsym}
\usepackage{caption}
\usepackage{subcaption}
\usepackage{amsmath}
\usepackage{multicol}
\usepackage{multirow}
\usepackage{mathtools}
\usepackage{enumitem}
\usepackage{setspace}
\usepackage{fancyhdr}
\usepackage{etoolbox}
\usepackage{bm}
\usepackage{tabularray}

\newcommand\doubleplus{\mathbin{+\mkern-5mu+}}
