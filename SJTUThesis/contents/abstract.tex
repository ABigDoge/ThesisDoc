% !TEX root = ../main.tex

\begin{abstract}[zh]
  编译器作为重要的系统软件,其效率与正确性一直是研究者们不断探讨的方向。
  随着形式化方法理论和工具的发展,编译器的形式化验证成为了确保其正确性的重要方法。
  静态单赋值(Static Single Assignment, SSA)形式作为一种中间语言(Intermediate Representation, IR)
  被现代主流编译器基础设施广泛采用,因为它能够实现许多基于数据流分析的编译优化。
  本文研究了如何将关键的函数式语言编译过程与静态单赋值中间语言通过形式化方法联系起来,
  从而在确保编译正确性的情况下使函数式语言的编译器能够充分利用静态单赋值的优势。
  现有的函数式编译器一般会将源程序编译为延续传递风格(Continuation Passing Style, CPS),
  从而得到明确的控制流。
  在本文中,我们设计了从延续传递风格的函数式程序到静态单赋值中间语言的转换算法,
  并对该转换算法进行了基于模拟技术的形式化验证。
  建立了这样的联系,才能够使经验证的函数式编译器复用基于静态单赋值的程序分析与编译优化,
  从而实现函数式编译器可靠性与效率的双重保障。

  本文还以代表性的基础函数式编程语言PCF(Programming Computable Functions)为研究对象,
  应用该经验证的编译过程构建出了PCF语言到LLVM中间语言的函数式编译器。
  具体而言,首先需要将直接风格的PCF源程序转换为延续传递风格,再编译到静态单赋值中间语言。
  我们对这条编译链的正确性进行形式化验证,并将它与LLVM中间语言连接起来,
  从而为构建经验证的高性能、高可靠函数式编译器提供了基础。
  本文涉及的所有算法、定义和定理证明均在Coq定理证明器中实现。
\end{abstract}

\begin{abstract}[en]
  As essential components of system software, compilers have been the 
  subject of continuous exploration by researchers regarding both 
  efficiency and correctness. With the development of the theory and tools
  for formal methods, formal verification of compilers has emerged as a 
  crucial approach to ensuring their correctness. 
  Static Single Assignment (SSA) form, widely adopted as an intermediate representation
  in modern mainstream compiler infrastructures, facilitates 
  numerous compilation optimizations based on data-flow analysis. 
  This paper investigates how to formally connect the key processes 
  of functional language compilation with SSA intermediate language, 
  enabling functional language compilers to harness the advantages of SSA 
  while ensuring compilation correctness. 
  Existing functional compilers often translate source programs into 
  Continuation Passing Style (CPS) to achieve explicit control flow. 
  In this paper, we design a transformation algorithm from CPS functional programs 
  to SSA intermediate language and formally verify this transformation algorithm 
  using simulation methods. Establishing such a connection enables 
  verified functional compilers to leverage SSA-based program analysis 
  and compilation optimizations, ensuring both reliability and efficiency in 
  functional compiler implementations.
    
  Furthermore, this paper takes the representative foundational functional 
  programming language, PCF (Programming Computable Functions), 
  as the source language. Applying the verified compilation process, 
  we construct a functional compiler for translating PCF programs to 
  LLVM intermediate language. Specifically, it involves the initial transformation
  of the direct-style PCF source code into CPS, followed by compilation into the SSA. 
  We formally verify the correctness of this compilation chain and connect it to the LLVM IR. 
  This establishes the groundwork for constructing a verified, high-performance, 
  and highly reliable functional compiler. All the algorithms, definitions, 
  and theorem proofs involved in this paper are implemented within the Coq theorem prover.
\end{abstract}
