% !TEX root = ../main.tex

\chapter{转换算法的语义保存性质证明} \label{ch:verify}

\section{编译链验证框架概览} \label{sec:verifyoverview}

在上一节中,我们介绍了直接风格PCF程序的CPS转换以及CPS到SSA的转换算法。
本节中我们将讨论如何对以上转换算法的正确性进行形式化验证。
给定一个PCF程序$t_{pcf}$,首先通过$\mathcal{F}_{proc}$将其转换为CPS程序,
然后再使用$\mathcal{G}_{proc}$把CPS程序转换为SSA程序$t_{ssa}$。
将两步转换算法合并之后完整的转换函数表示为
\begin{equation}
Comp(t_{pcf}) = \mathcal{G}_{proc}(\mathcal{F}_{proc}(t_{pcf})).
\end{equation}
如第\ref{sec:compcertbackend}节中所介绍,转换算法的语义保存性质可由定理~\ref{trm:bterm}和定理~\ref{trm:bdiv}表示,
其中$t \Downarrow n$表示$t$会终止并返回$n$,$t \Uparrow$表示$t$会发散。
如果转换后得到的SSA程序终止并返回$n$,则PCF程序也终止并返回$n$。另一方面,如果SSA程序发散,则PCF程序也发散。

\begin{theorem}[程序终止行为的保存]\label{trm:bterm} 
    \begin{tabbing}
     \\
    \quad\=$\forall \; t_{pcf}\; t_{ssa}\; n,\; $\=\kill
    \>$\forall \; t_{pcf}\; t_{ssa}\; n,\; t_{ssa}\Downarrow n\; \wedge \; t_{ssa}=Comp(t_{pcf}) \Longrightarrow t_{pcf}\Downarrow n.$
    \end{tabbing}
  \end{theorem}
  
  \begin{theorem}[程序发散行为的保存]\label{trm:bdiv}
    \begin{tabbing}
      \\
    \quad\=\kill
    \>$\forall \; t_{pcf}\; t_{ssa},\; t_{ssa}\Uparrow\; \wedge \; t_{ssa}=Comp(t_{pcf})\Longrightarrow t_{pcf}\Uparrow.$
    \end{tabbing}
  \end{theorem}

我们通过使用第\ref{sec:compcertbackend}节中介绍的模拟技术来证明语义保存性质的相关定理。
首先,我们需要为每个转换步骤建立前向模拟,然后将前向模拟性质组合成为完整编译过程的前向模拟,
并利用目标语言的确定性将其转化为对安全程序的后向模拟。该证明依赖于这样一个性质,
即终止的PCF程序总是会返回一个值(即不会陷入停滞状态)。
由于本文中使用的是显式的变量命名方法,我们假设该属性成立并计划之后采用局部无名表示
(Locally Nameless Representation)的方法来解决证明该属性的问题。
最后,我们从后向模拟中得出语义保存特性。

那么,我们在接下来的两节中将对这两步转换的前向模拟性质进行证明。
这两步转换的形式化证明结构相似,不过前者是通过使用多步模拟进行程序内部执行步骤的模拟,
而后者是使用星形模拟进行程序内部执行步骤的模拟。

\section{CPS转换的前向模拟} \label{sec:cpsforward}

\subsection{直接风格与CPS程序状态关系的不变式}

我们已经在第\ref{sec:cpstrans}节中讨论了将直接风格的PCF程序转换为CPS形式的CPS转换算法。
直接风格的PCF语言和与CPS语言之间的程序状态的不变量$\sim$关系在图\ref{simrelationcps}中定义如下。

图~\ref{simrelationcps}中定义的$\sim$关系递归地关联了PCF程序和CPS程序的相应部分。
规则(2)、(6)和(8)根据程序状态中的上下文和应用于值的延续变量$k$定义了PCF和CPS的匹配状态。
规则(4)定义了函数调用状态的匹配关系,规则(9)定义了最终状态的匹配关系。其他规则根据子代码项的关系递归地定义了匹配状态。
例如,规则(1)定义了PCF中$\mathbf{let}$语句与CPS中相应的$\mathbf{letcont}$语句的匹配状态。
在PCF程序中,使用$t_1$的值在$t_2$中替代变量$x$。在CPS程序中,在$u_2$中通过续延$k$把值传递给CPS项$u_1$。
因此,这两组子代码项符合$\sim$关系应当是这条规则的先决条件。其他规则的递归定义与之类似。

\begin{figure}[htbp]
    \centering
    \setlength{\jot}{10pt}
    \begin{gather*}
    \tag{1} \displaystyle{\frac{\begin{matrix}
        c t x'=c t x_{seq}\; (\mathbf{ let }\; x=t_1\; \mathbf{ in }\; t_2)\; c t x \quad
        t_{c p s} = \mathbf { letcont }\; k\; x=u_1\; \mathbf{ in }\; u_2 \\
        (t_1, c t x') \sim (u_2, u p d a t e\; l_{c p s}\; (k \mapsto t_{c p s})) \\
        (t_2, ctx)\sim (u_1, u p d a t e\; l_{c p s}\; (k \mapsto t_{c p s})) \end{matrix}}
        {(\mathbf{ let }\; x=t_1\; \mathbf{ in }\; t_2, c t x) \sim (t_{c p s}, l_{c p s})}} \\
    \tag{2} \displaystyle{\frac{\begin{matrix}
        c t x'=c t x_{seq}\; (\mathbf{ let }\; x=t_1\; \mathbf{ in }\; t_2)\; c t x \quad
        l_{cps}\; k = \mathbf{letcont}\; k\; x = u_1\; \mathbf{in}\; u_2 \end{matrix}}
        {(v, ctx')\sim (k\; v, l_{cps})}} \\
    \tag{3} \displaystyle{\frac{\begin{matrix}
        c t x'=c t x_{seq}\; (\mathbf{ fix }\; f\; x\; t_1)\; c t x \quad
        t_{cps} = \mathbf{ letfun }\; f\; k\; x\; u_1\; \mathbf{ in }\; u_2  \\
        (t_2, c t x') \sim (u_2, u p d a t e\; l_{c p s}\; (k \mapsto t_{c p s})) \\
        (t_1, ctx)\sim (u_1, update\; l_{cps}\; (k \mapsto t_{cps})) \end{matrix}}
        {((\mathbf{ fix }\; f\; x\; t_1)\; t_2, ctx)\sim (t_{cps}, l_{c p s})}} \\
    \tag{4} \displaystyle{\frac{\begin{matrix}
        c t x'=c t x_{seq}\; (\mathbf{ fix }\; f\; x\; t)\; ctx \end{matrix}}
        {(v, ctx')\sim (f\; k\; v, l_{cps})}} \\
    \tag{5} \displaystyle{\frac{\begin{matrix}
        c t x'=c t x_{seq}\; (op\; t_1\; t_2)\; c t x \quad
        t_{op} = \mathbf{ letop }\; y = op\; x_1\; x_2\; \mathbf{in}\; (k\; y) \\
        t_{cps2} = \mathbf{ letcont }\; k_2\; x_2 = t_{op}\; \mathbf{in}\; u_2 \quad
        t_{cps1} = \mathbf{ letcont }\; k_1\; x_1 = t_{cps2}\; \mathbf{ in }\; u_1 \\
        (t_1, ctx')\sim (u_1, update\; l_{cps}\; (k_1 \mapsto t_{cps1})) \\
        (t_2, ctx')\sim (u_2, update\; l_{cps}\; (k_2 \mapsto t_{cps2})) \end{matrix}}
        {(op\; t_1\; t_2, ctx)\sim (t_{cps1}, l_{cps})}} \\
    \tag{6} \displaystyle{\frac{\begin{matrix}
        ctx'=ctx_{seq}\; (op\; t_1\; t_2)\; ctx \quad
        l_{cps}\; k = \mathbf{letcont}\; k_1\; x_1 = t\; \mathbf{in}\; u_1 \end{matrix}}
        {(v, ctx')\sim (k_1\; v, l_{cps})}} \\    
    \tag{7} \displaystyle{\frac{\begin{matrix}
        ctx'=ctx_{seq}\; (\mathbf{ ifz }\; t_1\; t_2\; t_3)\; ctx \quad
        t_{if} = \mathbf{ ifz }\; x\; u_2\; u_3 \\
        t_{cps} = \mathbf{letcont}\; k\; x = t_{if}\; \mathbf{in}\; u_1 \quad
        (t_1, ctx')\sim (u_1, update\; l_{cps}\; (k \mapsto t_{cps})) \\
        (t_2, ctx)\sim (u_2, l_{cps}) \quad (t_3, ctx)\sim (u_3, l_{cps}) \end{matrix}}
        {(\mathbf{ ifz }\; t_1\; t_2\; t_3, ctx)\sim (t_{cps}, l_{cps})}} \\
    \tag{8} \displaystyle{\frac{\begin{matrix}
        ctx'=ctx_{seq}\; (\mathbf{ ifz }\; t_1\; t_2\; t_3)\; ctx \quad
        l_{cps}\; k = \mathbf{letcont}\; k\; x = (\mathbf{ ifz }\; x\; u_2\; u_3)\; \mathbf{in}\; u_1 \end{matrix}}
        {(v, ctx')\sim (k\; v, l_{cps})}} \\
    \tag{9} \displaystyle{\frac{\begin{matrix}
        c t x' = ctx_{stop} \end{matrix}}
        {(v, ctx')\sim (k_{init}\; v, l_{cps})}} \\
    \end{gather*}
    \caption{PCF与CPS程序状态之间的$\sim$关系规则}\label{simrelationcps}
\end{figure}


\subsection{直接风格与CPS程序内部执行步骤的模拟}

对于任何PCF程序$t_{pcf}$和转换后得到的CPS程序$t_{cps}$,$\sim$关系应该在它们的
初始状态(Initial State)下成立。
它们的初始状态被定义为
\begin{equation}
\mathtt{initial}(t_{pcf}) = (t_{pcf}, ctx_{stop}),\quad
\mathtt{initial}(t_{cps}) = (t_{cps}, loc_{empty}).
\end{equation}

在程序内部执行步骤的模拟中,我们使用多步模拟(Plus Simulation)来关联PCF和CPS程序状态的转换过程,
而不必担心无限驻留(Infinite Stuttering)问题。正如定理~\ref{def:fsimcps2}所示,
当PCF程序的状态在一步后到达$S'_{pcf}$,$S_{cps}$经过一步或多步转换到达$S'_{cps}$,
且它们抵达的新的状态仍然符合$\sim$关系。

\begin{theorem}[CPS转换中初始状态的模拟]\label{def:fsimcps}
    \begin{tabbing}
      \\
    \quad\=\kill 
    \>$\forall\; t_{pcf} \; t_{cps},\;
       t_{cps}=\mathcal{F}_{proc}(t_{pcf})\Longrightarrow \mathtt{initial}\; (t_{pcf})
       \sim \mathtt{initial}\; (t_{cps}).$
    \end{tabbing}
\end{theorem}

\begin{theorem}[CPS转换中程序内部执行步骤的模拟]\label{def:fsimcps2}
    \begin{tabbing}
      \\
    \quad\=\qquad\=$\exists\; S'_{cps},\; $\=\kill
    \>$\forall \; S_{pcf}\; S_{cps}\; S'_{pcf},\; S_{pcf}\rightarrow S'_{pcf}\; \wedge \; S_{pcf}\sim S_{cps} \Longrightarrow \exists\; S'_{cps},\; S'_{pcf}\sim S'_{cps}\; \wedge
        S_{cps}\xrightarrow{+} S'_{cps}$.
    \end{tabbing}
\end{theorem}

利用定理~\ref{def:fsimcps}和定理~\ref{def:fsimcps2},我们就可以证明出CPS转换得到的
CPS程序会有与直接风格PCF程序相同的可观察行为(Observable Behaviors),
也就是定理~\ref{trm:cpsbhvpt}和定理~\ref{trm:cpsbhvpd}。

\begin{theorem}[CPS程序对PCF程序终止行为的保存]\label{trm:cpsbhvpt} 
    \begin{tabbing}
     \\
    \quad\=$\forall \; t_{pcf}\; t_{cps}\; n,\; $\=\kill
    \>$\forall \; t_{pcf}\; t_{cps}\; n,\; t_{pcf}\Downarrow n\; \wedge \; t_{cps}=\mathcal{F}_{proc}(t_{pcf}) \Longrightarrow t_{cps}\Downarrow n.$
    \end{tabbing}
  \end{theorem}
  
  \begin{theorem}[CPS程序对PCF程序发散行为的保存]\label{trm:cpsbhvpd}
    \begin{tabbing}
      \\
    \quad\=\kill
    \>$\forall \; t_{pcf}\; t_{cps},\; t_{pcf}\Uparrow\; \wedge \; t_{cps}=\mathcal{F}_{proc}(t_{pcf})\Longrightarrow t_{cps}\Uparrow.$
    \end{tabbing}
  \end{theorem}


\section{CPS到SSA转换的前向模拟} \label{sec:cpsssaforward}

\subsection{CPS与SSA程序状态关系的不变式}

为了定义CPS到SSA转换的前向模拟,我们需要定义CPS和SSA语言程序状态之间的不变式,我们将其表示为$S_{cps} \sim S_{ssa}$。
它递归地将CPS程序的每个子项与生成的SSA程序中相应的代码段进行匹配。
例如,由$\mathbf{letcont}$代码项引入的局部延续$k$对应着SSA程序中名为$k$的基本代码块。
延续$k$的延续体与从基本块$k$开始的一部分SSA代码相关联。
当局部延续$k$应用于变量$x$时,相应的SSA指令将$x$赋值给与$k$的绑定变量同名的变量,并跳转到名为$k$的基本块。

$\sim$关系的规则定义如图~\ref{fig:simrelation}所示。规则(1)和(3)定义了CPS中将延续应用到某个值与
SSA程序中赋值、终止指令之间的匹配关系。规则(2)和(4)说明如果CPS程序包含类似$t$和$u$的子代码项,
则子代码项与相应位置的SSA代码片段应递归地相关联。规则(5)定义了函数调用时的状态匹配关系。
规则(6)和(7)展示了赋值表达式及二元算术运算表达式的匹配关系。
规则(8)定义了与条件语句的子代码项$t_1$和$t_2$相关联的特定基本块的匹配状态。

\begin{figure}[htbp]
    \centering
    % \small
    \setlength{\jot}{10pt}
    \begin{gather*}
        \tag{1} \displaystyle{\frac{\begin{matrix}
            loc_{cps}\; k = \textcolor{blue}{\mathbf{letcont}\; k\; x_1=t\; \mathbf{in}\; u}\\
            \mathbf{code}_{at}\; pc\; =\; \textcolor{purple}{x_1 = x}\quad
            \mathbf{code}_{at}\; (pc+1)\; =\; \textcolor{purple}{\mathbf{br_{uc}}\; k} \end{matrix}}
            {(\textcolor{blue}{k\; x},loc_{cps})\sim (t_{ssa},pc,ppc,loc_{ssa}\; x_1\mapsto x,s_{ssa})}}  \\
        \tag{2} \displaystyle{\frac{\begin{matrix}
            t_{cps} = \textcolor{blue}{\mathbf{letcont}\; k\; x_1=t\; \mathbf{in}\; u}\quad
            (u,loc_{cps})\sim (t_{ssa},pc,ppc,loc_{ssa},s_{ssa}) \\
            (t,loc_{cps}\; k\mapsto t_{cps})\sim (t_{ssa},(pc.l_f,k,0),pc,loc_{ssa},s_{ssa})\end{matrix}}
            {(t_{cps},loc_{cps}\; k\mapsto t_{cps})\sim (t_{ssa},pc,ppc,loc_{ssa},s_{ssa})}} \\
        \tag{3} \displaystyle{\frac{\begin{matrix}
            loc_{cps}\; k = \textcolor{blue}{\mathbf{letfun}\; f\; k\; x_1=t\; \mathbf{in}\; u}\;
            \mathtt{or}\; k=k_{init}\\
            \mathbf{code}_{at}\; pc\; =\; \textcolor{purple}{r_b = x}\quad 
            \mathbf{code}_{at}\; (pc+1)\; =\; \textcolor{purple}{\mathbf{ret}\; r_b} \end{matrix}}
            {(\textcolor{blue}{k\; x},loc_{cps})\sim (t_{ssa},pc,ppc,loc_{ssa}\; r_b\mapsto x,\mathbf{pop}\; s_{ssa})}}  \\
        \tag{4} \displaystyle{\frac{\begin{matrix}
            t_{cps} = \textcolor{blue}{\mathbf{letfun}\; f\; k\; x_1=t\; \mathbf{in}\; u} \\
            (t,loc_{cps}\; k\mapsto t_{cps})\sim (t_{ssa},(f,b_1,0),pc,loc_{ssa},s_{ssa}) \\
            (u,loc_{cps}\; k\mapsto t_{cps})\sim (t_{ssa},(main,b_1,0),pc,loc_{ssa},s_{ssa})\end{matrix}}
            {(t_{cps},loc_{cps}\; k\mapsto t_{cps})\sim (t_{ssa},pc,ppc,loc_{ssa},s_{ssa})}} \\
        \tag{5} \displaystyle{\frac{\begin{matrix}
            loc_{cps}\; k = \textcolor{blue}{\mathbf{letcont}\; k\; x_1=t\; \mathbf{in}\; u}\\
            \mathbf{code}_{at}\; pc\; =\; \textcolor{purple}{x_1 = \mathbf{call}\; f\; x}\quad
            \mathbf{code}_{at}\; (pc+1)\; =\; \textcolor{purple}{\mathbf{br_{uc}}\; k} \end{matrix}}
            {(\textcolor{blue}{f\; k\; x},loc_{cps})\sim (t_{ssa},pc,ppc,loc_{ssa},\mathbf{push}\; s_{ssa}\; pc)}}  \\
        \tag{6} \displaystyle{\frac{\begin{matrix}
            t_{cps} = \textcolor{blue}{\mathbf{letval}\; x=v\; \mathbf{in}\; t} \quad
            \mathbf{code}_{at}\; pc\; =\; \textcolor{purple}{x = v} \\
            (t,loc_{cps})\sim (t_{ssa},pc+1,ppc,loc_{ssa}\; x\mapsto v,s_{ssa}) \end{matrix}}
            {(t_{cps},loc_{cps})\sim (t_{ssa},pc,ppc,loc_{ssa},s_{ssa})}}  \\
        \tag{7} \displaystyle{\frac{\begin{matrix}
            t_{cps} = \textcolor{blue}{\mathbf{letop}\; x=op\; v_1\; v_2\; \mathbf{in}\; t} \quad
            \mathbf{code}_{at}\; pc\; =\; \textcolor{purple}{x = op\; v_1\; v_2} \\
            (t,loc_{cps})\sim (t_{ssa},pc+1,ppc,loc_{ssa}\; x\mapsto (\mathbf{eval}_{exp}\; v_1\; v_2),s_{ssa}) \end{matrix}}
            {(t_{cps},loc_{cps})\sim (t_{ssa},pc,ppc,loc_{ssa},s_{ssa})}}  \\ 
        \tag{8} \displaystyle{\frac{\begin{matrix}
            t_{cps} = \textcolor{blue}{\mathbf{ifz}\; x\; t_1\; t_2} \quad
            \mathbf{code}_{at}\; pc\; =\; \textcolor{purple}{\mathbf{icmp}\; x\; 0} \quad
            \mathbf{code}_{at}\; (pc+1)\; =\; \textcolor{purple}{\mathbf{br_c}\; b_n\; t_n\; f_n} \\
            (t_1,loc_{cps})\sim (t_{ssa},(pc.l_f, t_n, 0),pc,loc_{ssa},s_{ssa}) \\
            (t_2,loc_{cps})\sim (t_{ssa},(pc.l_f, f_n, 0),pc,loc_{ssa},s_{ssa}) \end{matrix}}
            {(t_{cps},loc_{cps})\sim (t_{ssa},pc,ppc,loc_{ssa},s_{ssa})}}  \\          
    \end{gather*}
    % \vspace{-0.4cm}
    \caption{CPS与SSA程序状态之间的$\sim$关系规则}\label{fig:simrelation}
\end{figure}

\subsection{度量函数的设计}

如第\ref{sec:verifyoverview}节中所述,我们使用在第\ref{sec:compcertbackend}节中
介绍的星形模拟来关联CPS和SSA程序内部执行时的状态。当CPS程序处理到$\mathbf{letcont}$代码项时,
可能会出现无限驻留问题:CPS程序需要进行一步转换,而目标SSA程序则不需要转换,就能和转换后的CPS程序状态匹配。
为了解决无限驻留问题,我们需要为源程序的状态定义一种度量函数$M$,在源程序执行过程中,
如果遇到目标程序驻留的情况,该度量函数需严格递减。
在本文中,我们使用$\mathbf{letcont}$结构的数量作为该度量,如图\ref{fig:measure}所示。
其实,从直觉上也可以感知到,只要CPS程序代码项中$\mathbf{letcont}$结构数量不是无限多,就不会出现无限驻留的情况。

\begin{figure}[htbp]
    \centering
    \begin{equation}
    \nonumber
        \begin{aligned}
            & M(t_{cps}) = \mathtt{match}\; t_{cps}\; \mathtt{with} \\
            & \quad |\; \mathbf{letcont}\; k\; x=t_1\; \mathbf{in}\; t_2 \Rightarrow 1 + M(t_1) + M(t_2) \\
            & \quad |\; \mathbf{ifz}\; v\; t_1\; t_2,\; \mathbf{letfun}\; f\; k\; x=t_1\; \mathbf{in}\; t_2 \Rightarrow M(t_1) + M(t_2) \\
            & \quad |\; \mathbf{letval}\; x=v\; \mathbf{in}\; t,\; \mathbf{letop}\; x=op\; x_1\; x_2\; \mathbf{in}\; t \Rightarrow M(t)\\
            & \quad |\; k\; v,\; f\; k\; v \Rightarrow 0 \\
        \end{aligned}
    \end{equation}
    \caption{CPS到SSA转换中的度量函数}\label{fig:measure}
\end{figure}


\subsection{CPS与SSA程序内部执行步骤的模拟}

对于任何CPS程序$t_{cps}$及其经过转换得到的SSA程序$t_{ssa}$,我们定义如果它们的初始状态满足
$\sim$关系并在整个程序执行过程中仍然能保持匹配,那么$t_{cps}$与$t_{ssa}$符合前向模拟性质。
在上一节中,我们已经定义了CPS程序的初始状态。这里我们定义SSA程序的初始状态为
\begin{equation}
\mathtt{initial}(t_{ssa}) = (t_{ssa}, (main, empty, 0), (main, empty, 0),loc_{empty}, s_{empty}).
\end{equation}
即$pc$指向主函数起始位置,映射$loc$和栈$s$都是空的。
定理\ref{def:fsimssa}说明了在程序初始状态时$\sim$关系成立,我们直观地对该定理进行定义展开即可证明。
定理\ref{trm:simustep2}说明了$\sim$关系在程序内部执行步骤中仍然成立。

我们通过对$S_{cps}\rightarrow S'_{cps}$的一步转换进行归纳来证明定理~\ref{trm:simustep2}。
在$S_{cps}\rightarrow S'_{cps}$的每一种情况下,也就是对于每一个子目标,
我们构造一个与$S'_{cps}$满足$\sim$关系的$S'_{ssa}$,
且能够从$S_{ssa}$经过若干步转换到达$S'_{ssa}$。
当然,我们需要知道经过SSA目标语言的哪些小步操作语义步骤能够到达$S'_{ssa}$。
当目标程序不进行转换就已经满足$S'_{cps}\sim S_{ssa}$的情况下,程序发生了驻留,
我们需要证明$M(S'_{cps})<M(S_{cps})$来确保不会发生无限驻留。

\begin{theorem}[CPS到SSA转换中初始状态的模拟]\label{def:fsimssa}
    \begin{tabbing}
      \\
        \quad\=\kill 
        \>$\forall\; t_{cps} \; t_{ssa},\;
        t_{ssa}=\mathcal{G}_{proc}(t_{cps})\Longrightarrow \mathtt{initial}\; (t_{cps})
        \sim \mathtt{initial}\; (t_{ssa}).$
    \end{tabbing}
  \end{theorem}

  \begin{theorem}[CPS到SSA转换中程序内部执行步骤的模拟]\label{trm:simustep2}
    \begin{tabbing}
      \\
    \quad\=\qquad\=$\exists\; S'_{ssa},\; $\=\kill
    \>$\forall \; S_{cps}\; S_{ssa}\; S'_{cps},\; S_{cps}\rightarrow S'_{cps}\; \wedge \; S_{cps}\sim S_{ssa} \Longrightarrow \exists\; S'_{ssa},\; S'_{cps}\sim S'_{ssa}\; \wedge$\\
    \>\>$(S_{ssa}\xrightarrow{+} S'_{ssa} \lor  (S_{ssa}\xrightarrow{*} S'_{ssa}\; \wedge \;  M(S'_{cps})<M(S_{cps})))$.
    \end{tabbing}
  \end{theorem}

为了理解在前向模拟证明中程序内部执行步骤的模拟是如何工作的,我们以图~\ref{factcps}和图~\ref{factssa}中计算阶乘的CPS和SSA程序为例。
一开始,它们的初始状态通过$\sim$相关联。根据它们的小步操作语义,下一个程序状态是:

\begin{itemize}
    \item $S_{cps1}:\; (\mathbf{letval}\; x_5=2\; \mathbf{in}\dots,\, loc_{empty}\; [k\mapsto fact_{cps}])$
    \item $S_{ssa1}:\; (fact_{ssa},\, (main, b_1, 0),\, (main, empty, 0),\, loc_{empty}\; [x_5\mapsto 2],\, s_{empty})$
\end{itemize}

为了证明$S_{cps1}\sim S_{ssa1}$成立,我们提取$(main, b_1, 0)$位置处的指令$[x_5 = 2;]$。
$S_{cps1}$和$S_{ssa1}$中更新后的映射分别被命名为$loc_{cps1}$和$loc_{ssa1}$。
经过一步转换之后,CPS和SSA程序的状态变为:

\begin{itemize}
    \item $S_{cps2}:\; (\mathbf{letcont}\; k_2\; y=k_{init}\; y\; \mathbf{in}\dots, loc_{cps1})$
    \item $S_{ssa2}:\; (fact_{ssa},\, (main, b_1, 1),\, (main, empty, 0),\, loc_{ssa1},\, s_{empty})$
\end{itemize}

根据图~\ref{fig:simrelation}中的规则(2),我们需要把子代码项$(k_{init}\; y)$和$(fact\; k_2\; x_5)$
分别与相应的SSA代码片段使用$\sim$关系匹配起来。

\begin{itemize}
    \item $(k_{init}\; y,\, loc_{cps1}\; [k_2\mapsto t_{cps2}])\sim $ \\ $(fact_{ssa},\, (main, b_1, 1),\, (main, empty, 0),\, loc_{ssa1},\; s_{empty})$
    \item $ (fact\; k_2\; x_5,\, loc_{cps1}\; [k_2\mapsto t_{cps2}])\sim $ \\ $(fact_{ssa},\, (main, k_2, 0),\, (main, b_1,1),\, loc_{ssa1},\, s_{empty})$
\end{itemize}

第一条匹配关系可以由规则(3)建立,而第二条匹配关系则可以由规则(5)建立。

如以上证明过程所示,我们可以一步步将CPS程序和SSA程序进行转换,并始终能找到它们符合$\sim$
关系的中间状态。最终,就可以得到结论,它们都会终止并返回相同的值。

与第\ref{sec:cpsforward}节中类似,利用定理~\ref{def:fsimcps}和定理~\ref{def:fsimcps2},我们就可以证明出转换得到的
SSA程序会与CPS程序具有相同的行为。如果CPS程序终止并返回$n$,转换后得到的SSA程序也终止并返回$n$。
如果CPS程序发散,转换后得到的SSA程序也发散。
即定理~\ref{trm:ssabhvpt}和定理~\ref{trm:ssabhvpd}。

\begin{theorem}[SSA程序对CPS程序终止行为的保存]\label{trm:ssabhvpt} 
    \begin{tabbing}
     \\
    \quad\=$\forall \; t_{cps}\; t_{ssa}\; n,\; $\=\kill
    \>$\forall \; t_{cps}\; t_{ssa}\; n,\; t_{cps}\Downarrow n\; \wedge \; t_{ssa}=\mathcal{G}_{proc}(t_{cps}) \Longrightarrow t_{ssa}\Downarrow n.$
    \end{tabbing}
  \end{theorem}
  
  \begin{theorem}[SSA程序对CPS程序发散行为的保存]\label{trm:ssabhvpd}
    \begin{tabbing}
      \\
    \quad\=\kill
    \>$\forall \; t_{cps}\; t_{ssa},\; t_{cps}\Uparrow\; \wedge \; t_{ssa}=\mathcal{G}_{proc}(t_{cps})\Longrightarrow t_{ssa}\Uparrow.$
    \end{tabbing}
  \end{theorem}

\section{组合与后向模拟}

将以上两节中得到的前向模拟性质相关定理进行组合,就可以得到整个编译过程的前向模拟。
组合定理~\ref{trm:cpsbhvpt}和定理~\ref{trm:ssabhvpt},即可证明定理~\ref{trm:topt}。
组合定理~\ref{trm:cpsbhvpd}和定理~\ref{trm:ssabhvpd},即可证明定理~\ref{trm:topd}。
我们还证明了目标SSA语言的确定性,即定理~\ref{trm:ssadeter}。

  \begin{theorem}[SSA程序对PCF程序终止行为的保存]\label{trm:topt} 
    \begin{tabbing}
     \\
    \quad\=$\forall \; t_{pcf}\; t_{ssa}\; n,\; $\=\kill
    \>$\forall \; t_{pcf}\; t_{ssa}\; n,\; t_{pcf}\Downarrow n\; \wedge \; t_{ssa}=\mathcal{G}_{proc}(t_{pcf}) \Longrightarrow t_{ssa}\Downarrow n.$
    \end{tabbing}
  \end{theorem}
  
  \begin{theorem}[SSA程序对PCF程序发散行为的保存]\label{trm:topd}
    \begin{tabbing}
      \\
    \quad\=\kill
    \>$\forall \; t_{pcf}\; t_{ssa},\; t_{pcf}\Uparrow\; \wedge \; t_{ssa}=\mathcal{G}_{proc}(t_{pcf})\Longrightarrow t_{ssa}\Uparrow.$
    \end{tabbing}
  \end{theorem}  

  \begin{theorem}[SSA语言的确定性]\label{trm:ssadeter} 
    \begin{tabbing}
     \\
    \quad\=$\forall \; t_{ssa}\; n_1\; n_2,\; $\=\kill
    \>$t_{ssa}\Downarrow n_1,\; t_{ssa}\Downarrow n_2 \Longrightarrow n_1 = n_2.$
    \end{tabbing}
  \end{theorem}

根据PCF到SSA编译过程的前向模拟及SSA目标语言的确定性,可以证明后向模拟,即定理~\ref{trm:bterm}和定理~\ref{trm:bdiv}。

对于定理~\ref{trm:bterm},首先假设PCF源程序不能终止,由定理~\ref{trm:topd}可知编译得到的SSA目标程序也不能终止。
但是,在定理~\ref{trm:bterm}的条件中,$t_{ssa}\Downarrow n$,SSA目标程序是终止的。
这与假设产生了矛盾,说明原假设错误,PCF源程序能够终止。
先将PCF源程序返回的结果值记为$m$,由定理~\ref{trm:topt}可知SSA目标程序返回的结果也是$m$。
由于SSA语言具有确定性,对于一个SSA程序,它的返回值是固定的。
所以,$m=n$,PCF源程序返回的结果值也是$n$,定理~\ref{trm:bterm}得证。

对于定理~\ref{trm:bdiv},假设源程序能够终止,
就存在某个值$n$是该PCF程序的返回值。由定理~\ref{trm:topt}可知,编译后得到的SSA目标程序也会终止并返回$n$。
这与定理~\ref{trm:bdiv}条件中的$t_{ssa}\Uparrow$相矛盾,说明原假设错误,PCF程序是发散的,定理~\ref{trm:bdiv}得证。

证明了完整编译过程对安全程序的反向模拟,就能够说明该转换算法具有语义保存性质,
转换得到的SSA程序保存了PCF源程序的语义。
