% !TEX root = ../main.tex

\chapter{CPS到SSA中间语言的转换算法} \label{ch:trans}

\section{源语言:CPS形式的PCF语言} \label{sec:cps}

PCF(Programming Computable Functions)是一种被广泛应用于研究的函数式语言~\cite{plotkin1977lcf}。
尽管PCF是一种小规模的语言,但它是图灵完备的,即所有可计算的函数都可以表示为PCF程序。
把它作为该编译框架的源语言,不会使算法和验证过于复杂而无法完成,也能够说明关键问题。
在本节中,我们定义了直接风格和CPS形式PCF语言的语法和小步操作语义。
我们还展示了将直接风格PCF转换为CPS形式的算法。

\subsection{源语言的语法和语义}

本文中介绍的PCF语言来自Dowek和L{'e}vy的工作~\cite{dowek2010introduction}。
它包括基本的$\lambda$-演算、算术运算表达式、条件表达式和不动点(Fixed Point)。PCF及其CPS形式的语法定义如图~\ref{pcfsyntax}所示。

\begin{figure}[htbp]
        \centering
        \begin{subfigure}[b]{0.4\textwidth}
            \flushright
        \begin{equation}
            \nonumber
            \begin{aligned}
                op\, := &\; +\; |\; -\; | \; \times \; |\; \div \\
                t\, := &\; i\; |\; x\; |\; t_1\; t_2\; |\; \mathbf{ifz}\; t_1\; t_2\; t_3 \\
                & |\; op\; t_1\; t_2 \\
                & |\; \mathbf{let}\; x=t_1\; \mathbf{in}\; t_2 \\
                & |\; \mathbf{fix}\; f\; x\; t
            \end{aligned}
        \end{equation}
        \caption{直接风格的PCF语言语法}\label{directpcf}
        \end{subfigure}
       % \hfill
        \begin{subfigure}[b]{0.5\textwidth}
            \flushleft
        \begin{equation}
            \nonumber
            \begin{aligned}
                v\, := &\; i\; |\; x \\
                t\, := &\; \mathbf{letval}\; x=v\; \mathbf{in}\; t\; |\; k\; v\; |\; \mathbf{ifz}\; v\; t_1\; t_2 \\
                & |\; f\; k\; v\; |\; \mathbf{letop}\; x=op\; x_1\; x_2\; \mathbf{in}\; t \\
                & |\; \mathbf{letcont}\; k\; x=t_1\; \mathbf{in}\; t_2 \\
                & |\; \mathbf{letfun}\; f\; k\; x=t_1\; \mathbf{in}\; t_2 
            \end{aligned}
        \end{equation}
        \caption{CPS形式的PCF语言语法}\label{cpspcf}
    \end{subfigure}
    \caption{PCF语言语法}\label{pcfsyntax}
    \end{figure}

最基本的PCF代码项(Term)包含自然数$i$和变量$x$。将代码项$t_1$应用到代码项$t_2$上表示为$t_1\; t_2$。
我们使用$\mathbf{let}\; x = t_1\; \mathbf{in}\; t_2$表示在$t_2$中将变量$x$替换为$t_1$的值。
PCF中的条件表达式记作$\mathbf{ifz}\; t_1\; t_2\; t_3$。如果$t_1$的值为0,则整个代码项规约到$t_2$。
否则,它将规约到$t_3$。PCF中的不动点代码项是$\mathbf{fix}\; f\; x\; t$,其中$t$中可能出现$f$。
例如,图~\ref{factpcf}中展示了使用直接风格的PCF实现阶乘功能并应用到参数2的程序。

CPS形式的PCF语言遵循Kennedy提出的CPS风格函数式程序结构~\cite{kennedy2007compiling}。
值$v$在CPS项中可以通过$\mathbf{letval}\; x = v\; \mathbf{in}\; t$语句引入。$k\; v$将延续(Continuation)$k$应用于参数$v$。
$f\; k\; v$将函数$f$应用于参数$v$,并传递延续变量$k$以接受此调用返回的结果。
语句$\mathbf{letop}\; x = op\; x_1\; x_2\; \mathbf{in}\; t$将变量$x$在项$t$中绑定为该二元运算的结果。
通过$\mathbf{letcont}\; k\; x = t_1\; \mathbf{in}\; t_2$引入局部延续(Local Continuation)$k$,其中$t_1$是延续$k$的延续体(Body)。
$\mathbf{letfun}\; f\; k\; x = t_1\; \mathbf{in}\; t_2$构造一个返回延续(Return Continuation)为$k$的函数$f$。
图~\ref{factcps}描述了CPS形式的PCF语言中阶乘函数的实现及对参数2的应用,其中每个计算步骤都是被显式命名的。
可以看到,顶层延续变量$k_{init}$接受整个程序返回的结果,并由其上下文绑定。

\begin{figure}[htbp]
        \centering
        \begin{subfigure}[b]{0.3\textwidth}
            \flushright
        % \small
        \begin{equation}
            \nonumber
            \begin{aligned}
            & (\mathbf{fix}\; fact\; x \\
            & \quad \mathbf{ifz}\; x\; 1 \\
            & \quad\quad (x*(fact\; (x-1)))) \\
            & 2
            \end{aligned}
        \end{equation}
        \caption{直接风格的PCF阶乘程序}\label{factpcf}
        \end{subfigure}
        \begin{subfigure}[b]{0.68\textwidth}
            \flushleft
            % \small
        \begin{equation}
            \nonumber
            \begin{aligned}
            & \mathbf{letfun}\; fact\; k\; x = (\mathbf{ifz}\; x\; (\mathbf{letval}\; x_1=1\; \mathbf{in}\; (k\; x_1))\\
            & \quad (\mathbf{letval}\; x_2=1\; \mathbf{in}\; (\mathbf{letop}\; x_4=x-x_2\; \mathbf{in} \\
            & \quad\quad \mathbf{letcont}\; k_1\; z= (\mathbf{letop}\; x_3=x*z\; \mathbf{in}\; (k\; x_3))\\
            & \quad\quad\quad  \mathbf{in}\; fact\; k_1\; x_4)))\; \mathbf{in} \\
            & (\mathbf{letval}\; x_5=2\; \mathbf{in}\; (\mathbf{letcont}\; k_2\; y=k_{init}\; y\; \mathbf{in}\; (fact\; k_2\; x_5))) \\
            \end{aligned}
        \end{equation}
        \caption{CPS形式的PCF阶乘程序}\label{factcps}
        \end{subfigure}
    \caption{使用PCF语言实现阶乘程序}\label{factpcfcps}
    \end{figure}

接下来,我们需要为以上所介绍的PCF语言提供小步操作语义。
$S\rightarrow S'$表示执行一步操作可以使程序从初始状态$S$到达状态$S'$。
在直接风格的PCF程序中,程序状态之间的转换步骤表示为:
\begin{equation}
(t_{pcf},ctx)\rightarrow (t'_{pcf},ctx').
\end{equation}
$t_{pcf}$表示的是正在被求值(Evaluate)的表达式的PCF代码项。
$ctx$是一个包含代码项序列的上下文(Context),用于在当前$t_{pcf}$的求值完成后进行下一步执行操作。
当$ctx = ctx_{stop}$时,程序执行在$t_{pcf}$求值完成后即可结束。否则,当$ctx = ctx_{seq}\; t\; ctx$,
表示$t_{pcf}$的值将作为后续代码项$t$的参数,程序继续执行。经过一步转换之后,
新的程序状态包含了更新后的PCF代码项$t'_{pcf}$和新的上下文$ctx'$。转换规则如图~\ref{pcfopsem}所示。

$\mathbf{let}$表达式在$t_2$中用代码项$t_1$的值替换变量$x$。
因此,我们首先对$t_1$进行求值并将当前求值完成后下一步需要处理的代码项追加在上下文$ctx$中。
然后,在得到$t_1$的值后,我们将$t_2$中的变量$x$进行替换。当将不动点应用于代码项$t_2$时,
也是进行类似的操作。先将不动点放入上下文中,当$t_2$的值计算完毕后,将用它来替换不动点中的参数$x$,然后计算继续进行。
对于$\mathbf{ifz}$条件语句,我们求出$t_1$的值$n$,并根据$n$是否为0来确定下一步规约操作。

\begin{figure}[t]
    \centering
    \begin{subfigure}[t]{0.43\textwidth}
        \setlength{\jot}{10pt}
        \begin{gather*}
            \displaystyle{\frac{t_{pcf}=(\mathbf{let}\; x = t_1\; \mathbf{in}\; t_2)} {(t_{pcf},\; ctx)\rightarrow (t_1,\; ctx_{seq}\; t_{pcf}\; ctx)}} \\
            \displaystyle{\frac{t_1\; is\; a\; value\quad t_3=(\mathbf{let}\; x = t_1\; \mathbf{in}\; t_2)} {(t_1,\; ctx_{seq}\; t_3\; ctx)\rightarrow (t_2 [t_1/x],\; ctx)}} \\
            \displaystyle{\frac{v\; is\; a\; value\quad t_3=(\mathbf{ifz}\; t_1\; t_2\; t_3)} {(v,\; ctx_{seq}\; t_3\; ctx)\rightarrow (t_3 [v/t_1],\; ctx)}} \\
            \displaystyle{\frac{t_{pcf}=(\mathbf{ifz}\; t_1\; t_2\; t_3)} {(t_{pcf},\; ctx)\rightarrow (t_1,\; ctx_{seq}\; t_{pcf}\; ctx)}} \\
            \displaystyle{(\mathbf{ifz}\; 0\; t_2\; t_3,\; ctx)\rightarrow (t_2,\; ctx)} \\
            \displaystyle{\frac{n \neq 0}{(\mathbf{ifz}\; n\; t_2\; t_3,\; ctx)\rightarrow (t_3,\; ctx)}} \\
        \end{gather*}
    \end{subfigure}
    \begin{subfigure}[t]{0.55\textwidth}
        \setlength{\jot}{10pt}
        \begin{gather*}
            \displaystyle{\frac{t_{pcf}=((\mathbf{fix}\; f\; x\; t_1)\; t_2)} {(t_{pcf},\; ctx)\rightarrow (t_2,\; ctx_{seq}\; (\mathbf{fix}\; f\; x\; t_1)\; ctx)}} \\
            \displaystyle{\frac{t_2\; is\; a\; value\quad t_3=(\mathbf{fix}\; f\; x\; t_1)} {(t_2,\; ctx_{seq}\; t_3\; ctx)\rightarrow (t_1 [t_2/x,t_3/f],\; ctx)}} \\
            \displaystyle{(op\; t_1\; t_2,\; ctx)\rightarrow (t_1,\; ctx_{seq}\; (op\; t_1\; t_2,\; ctx)\; ctx)} \\
            \displaystyle{\frac{v_1\; is\; a\; value\quad t_3=(op\; t_1\; t_2,\; ctx)} {(v_1,\; ctx_{seq}\; t_3\; ctx)\rightarrow (t_2,\; ctx_{seq}\; (op\; v_1\; t_2)\; ctx)}} \\
            \displaystyle{\frac{v_1,\; v_2\; are\; values\quad n=\mathbf{eval}_{op}\; op\; v_1\; v_2}{(v_2,\; ctx_{seq}\; (op\; v_1\; t_2)\; ctx)\rightarrow (n,\; ctx)}} \\
        \end{gather*}
    \end{subfigure}   
    \caption{直接风格的PCF语言小步操作语义转换规则} \label{pcfopsem}
\end{figure}

CPS形式的PCF语言的小步操作语义表示的推导规则形如
\begin{equation}
(t_{cps},loc_{cps})\rightarrow(t'_{cps},loc'_{cps}).
\end{equation}
同样的,$t_{cps}$是正在被计算的CPS代码项,$loc_{cps}$是从延续变量名或函数名到引入它们的代码项的映射(Mapping)。
$t'_{cps}$和$loc'_{cps}$分别是更新后的CPS代码项和映射。
CPS形式的PCF语言的小步操作语义转换规则如图~\ref{cpsopsem}所示。
$\mathbf{letval}$表达式可以将新的值引入代码项$t$,我们只需要在$t$中用$v$替换变量$x$。
对于代码项$(\mathbf{letcont}\; k\; x=t_1\; \mathbf{in}\; t_2)$,
首先对$t_2$求值并更新$loc_{cps}$,添加从延续变量$k$到该代码项的映射。
我们使用$loc_{cps}\; [k\mapsto t_{cps}]$这种标记来表示对映射$loc_{cps}$的更新操作。
当把延续$k$应用于值$v$时(用$k\; v$表示),由于$t_1$是延续$k$的延续体,我们将$t_1$中的变量$x$替换为$v$。
$\mathbf{letfun}$代码项的规约与之类似,当我们遇到$\mathbf{letfun}$结构时,我们会更新从$f$到该项的映射。
当我们将延续$k_0$和变量$x_0$作为参数传递给$f$时,我们分别用$k_0$和$x_0$替换函数体$t_1$中的$k$和$x$。

\begin{figure}[htbp]
    \centering
    \begin{subfigure}[t]{0.43\textwidth}
        \setlength{\jot}{10pt}
        \begin{gather*}
            \displaystyle{\frac{t_{cps}=(\mathbf{letval}\; x=v\; \mathbf{in}\; t)} {(t_{cps},\; loc_{cps})\rightarrow (t [v/x],\; loc_{cps})}} \\
            \displaystyle{\frac{t_{cps}=(\mathbf{ifz}\; 0\; t_1\; t_2)} {(t_{cps},\; loc_{cps})\rightarrow (t_1,\; loc_{cps})}} \\
            \displaystyle{\frac{t_{cps}=(\mathbf{ifz}\; n\; t_1\; t_2)\quad n \neq 0} {(t_{cps},\; loc_{cps})\rightarrow (t_2,\; loc_{cps})}} \\
            \displaystyle{\frac{loc_{cps}\; k = (\mathbf{letcont}\; k\; x=t_1\; \mathbf{in}\; t_2)}{(k\; v,\; loc_{cps})\rightarrow (t_1 [v/x],\; loc_{cps})}} \\
        \end{gather*}
    \end{subfigure}
    \begin{subfigure}[t]{0.55\textwidth}
        \setlength{\jot}{10pt}
        \begin{gather*}
            \displaystyle{\frac{t_{cps}=(\mathbf{letop}\; x=op\; v_1\; v_2\; \mathbf{in}\; t)}
            {(t_{cps},\; loc_{cps})\rightarrow (t [(\mathbf{eval}_{op}\; op\; v_1\; v_2)/x],\; loc_{cps})}} \\
            \displaystyle{\frac{t_{cps}=(\mathbf{letfun}\; f\; k\; x=t_1\; \mathbf{in}\; t_2)} {(t_{cps},\; loc_{cps})\rightarrow (t_2,\; loc_{cps}\; [f\mapsto t_{cps}])}} \\
            \quad \displaystyle{\frac{loc_{cps}\; f = (\mathbf{letfun}\; f\; k\; x=t_1\; \mathbf{in}\; t_2)}{(f\; k_0\; x_0,\; loc_{cps})\rightarrow (t_1 [x_0/x, k_0/k],\; loc_{cps})}} \\
            \displaystyle{\frac{t_{cps}=(\mathbf{letcont}\; k\; x=t_1\; \mathbf{in}\; t_2)} {(t_{cps},\; loc_{cps})\rightarrow (t_2,\; loc_{cps}\; [k\mapsto t_{cps}])}} 
        \end{gather*}
    \end{subfigure}   
    \caption{CPS形式的PCF语言小步操作语义转换规则}\label{cpsopsem}
\end{figure}

\subsection{直接风格PCF语言的CPS转换} \label{sec:cpstrans}

如第\ref{sec:bg_cps}节中所述,函数式语言的编译器通常会将直接风格的源程序转换为CPS形式,
即进行CPS转换(CPS Transformation)。本节中,我们将介绍把PCF源程序转换为CPS形式的编译算法。
该算法遵循CPS转换的一般模式~\cite{plotkin1975call,danvy2007one},根据计算的层次顺序递归地解构处理PCF代码项。
例如,如果PCF代码项$t$的计算步骤可以有序地分为对$t_1$和$t_2$的计算,
那么我们就可以把对$t_2$进行转换得到的CPS代码项放入之后要处理的参数中,并在下一步中直接开始处理$t_1$。 

转换过程由图~\ref{algo:cpstrans}中的函数$\mathcal{F}_{proc}$描述,它使用了一个更加广义的转换函数$\mathcal{F}$。
函数$\mathcal{F}$的输入和输出分别表示为$(t_{pcf}, l_v, \kappa)$和$t_{cps}$。
$t_{pcf}$是待转换的PCF程序。$\kappa$的结构表示为$\lambda x. t'_{cps}$,
代表了当前代码项被归约到一个值后要被应用的CPS代码项(延续)。$t_{cps}$是其生成的CPS程序。
在我们的CPS转换算法中,变量使用有名字的字符串名称。$l_v$是已经被使用的变量列表,
新生成的名称不能与$l_v$中的名称冲突。例如,使用图~\ref{algo:cpstrans}中的规则(3)对$t_1\; t_2$进行转换,
生成的新变量$k,\, x,\, y,\, z$必须不能已经存在于$l_v$中。该步转换之后,它们被添加到$l_v$中。
为了简化转换规则,我们没有在图中的算法里写出使用$l_v$指定生成新变量的具体操作。
它所做的其实是通过维护最新生成的变量的下标,来选择最小未使用的正数作为新变量的下标。
在初始状态中,要转换的PCF程序即为源程序$t_{pcf}$,新生成的名称只有顶层延续变量的名字$k_{init}$。
此时参数$\kappa$为$\lambda x. (k_{init}\; x)$,表示它接受整个生成的$t_{cps}$程序的值。
因此,应用$k_{init}$的值将是程序的最终结果。

\begin{figure}[t]
    \centering
    % \small
    % \setlength{\jot}{10pt}
    \begin{algorithm}[H]
        \caption{CPS转换}
        \SetAlgoLined
        $\mathcal{F}_{proc}:\quad \mathbf{Input:}\; t_{pcf}\quad \mathbf{Output:}\; t_{cps}$\\
        $\mathcal{F}_{proc}(t_{pcf})\coloneqq \mathbf{\mathcal{F}}(t_{pcf}, [k_{init}], \lambda x. (k_{init}\; x))$\\
        \vspace*{0.5em}
        $\mathcal{F}:\quad \mathbf{Input:}\; t_{pcf},\; l_v,\; \kappa \quad \mathbf{Output:}\; t_{cps}$\\ 
        $(1).\; \mathbf{\mathcal{F}}(i, l_v, \kappa) \coloneqq \textcolor{blue}{\mathbf{letval}\; x= i\; \mathbf{in}\; \kappa(x)} $ \\
        $(2).\; \mathbf{\mathcal{F}}(x, l_v, \kappa) \coloneqq \kappa(x) $ \\
        $(3).\; \mathbf{\mathcal{F}}(t_1\; t_2, l_v, \kappa) \coloneqq \mathbf{\mathcal{F}}(t_1, l'_v,\lambda x. \mathbf{\mathcal{F}}(t_2, l'_v, \lambda y. (\textcolor{blue}{\mathbf{letcont}\; k\; z=\kappa(z)\; \mathbf{in}\; (x\; k\; y)})))$ \\
        $\quad\quad \mathtt{where}\; l'_v = l_v \doubleplus [k;x;y;z]$ \\
        $(4).\; \mathbf{\mathcal{F}}(\mathbf{ifz}\; t_1\; t_2\; t_3, l_v, \kappa) \coloneqq \mathbf{\mathcal{F}}(t_1, l'_v, \lambda x. (\textcolor{blue}{\mathbf{ifz}\; x\; \mathbf{\mathcal{F}}(t_2, l'_v, \kappa)\; \mathbf{\mathcal{F}}(t_3, l'_v, \kappa)}))  $ \\
        $\quad\quad \mathtt{where}\; l'_v = l_v \doubleplus [x]$ \\
        $(5).\; \mathbf{\mathcal{F}}(op\; t_1\; t_2, l_v, \kappa) \coloneqq \mathbf{\mathcal{F}}(t_1, l'_v,\lambda x.\mathbf{\mathcal{F}}(t_2, l'_v, \lambda y. (\textcolor{blue}{\mathbf{letop}\; z=op\; x\; y\; \mathbf{in}\; \kappa(z)}))) $ \\
        $\quad\quad \mathtt{where}\; l'_v = l_v \doubleplus [x;y;z]$ \\
        $(6).\; \mathbf{\mathcal{F}}(\mathbf{let}\; x=t_1\; \mathbf{in}\; t_2, l_v, \kappa) \coloneqq \mathbf{\mathcal{F}}(t_1, l_v, \lambda x. \mathbf{\mathcal{F}}(t_2, l_v, \kappa)) $ \\
        $(7).\; \mathbf{\mathcal{F}}(\mathbf{fix}\; f\; x\; t, l_v, \kappa) \coloneqq \textcolor{blue}{\mathbf{letfun}\; f\; k\; x= \mathbf{\mathcal{F}}(t, l'_v, \lambda y. (k\; y))\; \mathbf{in}\; \kappa(f)}$ \\
        $\quad\quad \mathtt{where}\; l'_v = l_v \doubleplus [k;y]$
    \end{algorithm}
    \caption{CPS转换算法}\label{algo:cpstrans}
\end{figure}

\section{目标语言:类似LLVM IR的SSA语言}

本文中使用的目标SSA语言是一种简化版本的LLVM IR,保留了其最基本的程序结构层次。
在本节接下来的内容中,我们将定义它的语法和小步操作语义,并讨论对SSA程序中自由变量的处理和避免。

\subsection{目标语言的语法和语义}

\begin{figure}[htbp]
    \centering
    \begin{equation}
        \nonumber
        \begin{aligned}
            l &\coloneqq string & v &\coloneqq i\; |\; x & r &\coloneqq \mathbf{ret}\; v\; |\; \mathbf{br_{uc}}\; l\; |\; \mathbf{br_c}\; v\; l_1\; l_2 \\
            \phi_a &\coloneqq (l,\; v) & \phi &\coloneqq x = \overline{\phi_a} &  c &\coloneqq v\; |\; op\; v_1\; v_2\; |\; \mathbf{icmp}\; v_1\; v_2\; |\; \mathbf{call}\; x\; v \\
            a &\coloneqq x = c; & b &\coloneqq l:\; \overline{\phi}\; \overline{a}\; r & f &\coloneqq \mathbf{define}\; l_1(l_2)\; \overline{b} \quad\quad\quad t \coloneqq \overline{f} \\
        \end{aligned}
    \end{equation}
    \caption{SSA目标语言的语法}\label{synssa}
\end{figure}

我们使用的目标SSA语言的语法定义如图~\ref{synssa}所示。
顶层的翻译单元(Translation Unit)$t$由一系列函数定义组成。
一个函数$f$包含函数名$l_1$、参数$l_2$和一系列基本代码块$\overline{b}$。
一个基本代码块$b$由它的标签$l$、$\overline{\phi}$节点序列、指令$\overline{a}$序列和一个终止指令(Terminator)$r$组成。
一条指令$a$将命令$c$的值赋给变量$x$。$c$包括值$v$、二元运算表达式、比较两个值是否相等的命令和函数调用。
终止指令包括$\mathbf{ret}$和$\mathbf{br}$,分别用于表示函数返回和块之间的跳转。
作为示例程序,图~\ref{factssa}展示了SSA中的阶乘函数与2的阶乘。

\begin{figure}[ht]
    \centering
    \begin{equation}
        \nonumber
        \begin{aligned}
            & \mathbf{define}\; fact\; (x)\\
            & \quad b_1:\; b_0 = \mathbf{icmp}\; x\; 0;\; \mathbf{br_c}\; b_0\; t_0\; f_0; \\
            & \quad t_0:\; x_1 = 1;\; r_{t0} = x_1;\; \mathbf{ret}\; r_{t0}; \\
            & \quad f_0:\; x_2 = 1;\; x_4 = x - x_2;\; z = \mathbf{call}\; fact\; x_4;\; \mathbf{br_{uc}}\; k_1; \\
            & \quad k_1:\; x_3 = x*z;\; r_{k1} = x_3;\; \mathbf{ret}\; r_{k1}; \\
            & \mathbf{define}\; main\; ( )\\
            & \quad b_1:\; x_5 = 2;\; y = \mathbf{call}\; fact\; x_5;\; \mathbf{br_{uc}}\; k_2;\\
            & \quad k_2:\; r_{k2} = y;\; \mathbf{ret}\; r_{k2};
        \end{aligned}
    \end{equation}
    \caption{SSA语言中的阶乘程序}\label{factssa}
\end{figure}

目标SSA语言的小步操作语义规则表示为
\begin{equation}
(pc, ppc, loc_{ssa}, s_{ssa}) \rightarrow (pc', ppc', loc'_{ssa}, s'_{ssa}).
\end{equation}
$pc$是程序计数器(Program Counter),用于定位当前指令,由三个元素$(l_{f}, l_b, n)$组合而成。
其中,$l_{f}$是当前函数标签,$l_b$是当前基本代码块标签,$n$表示基本块中指令的位置。
可以使用$\mathbf{code}_{at}\; pc$获取$pc$位置处的指令。$ppc$存储了上一个块中进行跳转之前的程序计数器值,
以便对$\phi$-节点进行求值。$loc_{ssa}$是变量名到它们的值的映射。
我们在堆栈$s_{ssa}$中保留调用当前函数的指令的程序计数器,以便在当前函数返回时能够返回到调用时的位置。

SSA语言的操作语义转换规则如图~\ref{ssaopsem}所示。第一条规则描述了函数调用的程序状态转换。
控制流跳转到函数$f$的开头,并将函数返回地址存储在$s_{ssa}$中。
$\mathbf{arg}\; f$表示获取函数$f$的参数,$loc_{ssa}\; [x\mapsto v_0]$表示把从$x$到$v_0$的映射添加到$loc_{ssa}$中。
例如,在图~\ref{factssa}中,当主函数调用$fact$函数时,按照图~\ref{ssaopsem}中的第一条规则,
程序状态从$((main,b_1,1),\, (main,empty,1),\, loc_{ssa},\, s_{ssa})$ 
转换到$((fact,b_1,0),\, (main,b_1,1),\, loc_{ssa}$ \\ $[x\mapsto 2],\, \mathbf{push}\; s_{ssa}\; (main,b_1,1))$。
第二条规则定义了当前函数返回并将返回值传递给调用语句时发生的状态转换。
它从$s_{ssa}$的顶部取出程序计数器$npc$,存储返回值,并跳回该调用语句的位置。

\begin{figure}[htbp]
    % \vspace*{-0.3cm}
    \centering
    % \small
    \setlength{\jot}{10pt}
    \begin{gather*}
        \displaystyle{\frac{\mathbf{code}_{at}\; pc\; =\; (y=\mathbf{call}\; f\; v_0)\quad \mathbf{arg}\; f = x}
        {(pc,\; ppc,\; loc_{ssa},\; s_{ssa})\rightarrow ((f,b_1,0),\; pc,\; loc_{ssa}\; [x\mapsto v_0],\; \mathbf{push}\; s_{ssa}\; pc )}} \\
        % & \displaystyle{\frac{\mathbf{code}_{at}\; pc\; =\; id = phi,\quad n = \mathbf{eval}_{phi}\ phi\ ppc\ loc_{ssa} } {(pc,\; ppc,\; loc_{ssa},\; ploc_{ssa})\rightarrow (pc+1,\; ppc,\; loc_{ssa}\; id\mapsto n,\; ploc_{ssa})}} \\
        \displaystyle{\frac{\begin{matrix}\mathbf{code}_{at}\; pc\; =\; (\mathbf{ret}\; v)\quad pc.l_f \neq main\quad npc = \mathbf{top}\; s_{ssa}
        \\ \mathbf{code}_{at}\; npc\; =\; (x=\mathbf{call}\; f\; v_0)\end{matrix}} 
        {(pc,\; ppc,\; loc_{ssa},\; s_{ssa})\rightarrow (npc+1,\; pc,\; loc_{ssa}\; [x\mapsto v],\; \mathbf{pop}\; s_{ssa})}} \\
        \displaystyle{\frac{\mathbf{code}_{at}\; pc\; =\; (x=\overline{\phi_a})\quad n\; =\; \mathbf{eval}_{\phi}\ \overline{\phi_a}\ ppc}
            {(pc,\; ppc,\; loc_{ssa},\; s_{ssa})\rightarrow (pc+1,\; ppc,\; loc_{ssa}\; [x\mapsto n],\; s_{ssa})}} \\
        \displaystyle{\frac{\mathbf{code}_{at}\; pc\; =\; (x=op\; v_1\; v_2)\quad n\; =\; \mathbf{eval}_{exp}\ loc_{ssa}\ op\; v_1\; v_2}
        {(pc,\; ppc,\; loc_{ssa},\; s_{ssa})\rightarrow (pc+1,\; ppc,\; loc_{ssa}\; [x\mapsto n],\; s_{ssa})}} \\
        \displaystyle{\frac{\mathbf{code}_{at}\; pc\; =\; (x=v)\quad n\; =\; v\; is\; value?\; v\; :\; (loc_{ssa}\; v)}
        {(pc,\; ppc,\; loc_{ssa},\; s_{ssa})\rightarrow (pc+1,\; ppc,\; loc_{ssa}\; [x\mapsto n],\; s_{ssa})}} \\
        \displaystyle{\frac{\mathbf{code}_{at}\; pc\; =\; (x=\mathbf{icmp}\; v_1\; v_2)\quad \mathbf{if}\; (\mathbf{equal}_{val}\; loc_{ssa}\; v_1\; v_2)\; n=1\; \mathbf{else}\; n=0}
        {(pc,\; ppc,\; loc_{ssa},\; s_{ssa})\rightarrow (pc+1,\; ppc,\; loc_{ssa}\; [x\mapsto n],\; s_{ssa})}} \\
        \displaystyle{\frac{\mathbf{code}_{at}\; pc\; =\; (\mathbf{br_{uc}}\; l)}
        {(pc,\; ppc,\; loc_{ssa},\; s_{ssa})\rightarrow ((pc.l_f, l, 0),\; pc,\; loc_{ssa},\; s_{ssa})}} \\
        \displaystyle{\frac{\mathbf{code}_{at}\; pc\; =\; (\mathbf{br_c}\; v\; l_1\; l_2)\quad \mathbf{if}\; (\mathbf{equal}_{val}\; loc_{ssa}\; v\; 0)\; l_n=l_1\; \mathbf{else}\; l_n=l_2}
        {(pc,\; ppc,\; loc_{ssa},\; s_{ssa})\rightarrow ((pc.l_f, l_n, 0),\; pc,\; loc_{ssa},\; s_{ssa})}}
    \end{gather*}
    \caption{目标SSA语言的小步操作语义转换规则}\label{ssaopsem}
\end{figure}

\subsection{含有自由变量的函数}

我们的目标SSA语言与LLVM IR之间的一个重要区别在于,$loc_{ssa}$是一个全局映射(Global Mapping),
我们允许函数包含未在该函数中定义的自由变量(Free Variables)。这样做是因为我们的源语言是带有自由变量的高阶函数。
典型的函数式语言编译器会对CPS程序使用闭包转换(Closure Conversion),将开放函数转换为闭包。
闭包转换的形式化验证已经得到广泛研究~\cite{paraskevopoulou2019closure,wang-esop2016}。
为简单起见,我们没有将闭包转换包含在我们的编译链中。我们的SSA语言支持闭包和开放函数,
并专注于CPS到SSA转换的本质。在接下来的内容中我们将讨论这一点。

\section{转换算法设计} \label{sec:cpsssatrans}

\begin{figure}[!ht]
    \centering
    % \setlength{\jot}{10pt}
    \begin{algorithm}[H]
        \caption{CPS$\rightarrow $SSA 转换}
        \SetAlgoLined
        $\mathcal{G}_{proc}:\quad \mathbf{Input:}\; t_{cps}\quad \mathbf{Output:}\; t_{ssa}$\\
        $\mathcal{G}_{proc}(t_{cps})\coloneqq t_{ssa}  $\\
        $\quad\quad \mathtt{where}\; (t_{ssa}, n, loc) = \mathbf{\mathcal{G}}(t_{cps}, \mathbf{app}_b\; nil\; main, (main, b_1, 0), 0, loc_{empty}) $\\
        % $\quad\quad\quad\quad t_{ssa}$ \\
        \vspace*{0.5em}
        $ \mathbf{\mathcal{G}}:\quad \mathbf{Input:}\; t_{cps}, t_{ssa}, pc, n, loc\quad \mathbf{Output:}\; t'_{ssa}, n', loc' $ \\
        $ (1).\; \mathbf{\mathcal{G}}(\textcolor{blue}{\mathbf{letval}\; x=v\; \mathbf{in}\; t}, t_{ssa}, pc, n, loc) \coloneqq  $ \\
        $ \quad\quad\quad \mathbf{\mathcal{G}}(t, \mathbf{app}_i\; t_{ssa}\; pc\; [\textcolor{purple}{x=v;}], pc+1, n, loc) $ \\
        $ (2).\; \mathbf{\mathcal{G}}(\textcolor{blue}{\mathbf{letop}\; x=op\; x_1\; x_2\; \mathbf{in}\; t}, t_{ssa}, pc, n, loc) \coloneqq  $ \\
        $ \quad\quad\quad \mathbf{\mathcal{G}}(t, \mathbf{app}_i\; t_{ssa}\; pc\; [\textcolor{purple}{x=op\; x_1\; x_2;}], pc+1, n, loc) $ \\
        $ (3).\; \mathbf{\mathcal{G}}(\textcolor{blue}{\mathbf{letfun}\; f\; k\; x=t_1\; \mathbf{in}\; t_2}, t_{ssa}, pc, n, loc) \coloneqq \mathbf{\mathcal{G}}(t_2, t', pc, n', loc') $ \\
        $ \quad\quad \mathtt{where}\; (t', n', loc') = \mathbf{\mathcal{G}}(t_1, \mathbf{app}_p\; t_{ssa}\; f, (f, b_1, 0), 0, loc\; (k\mapsto\; t_{cps})) $ \\
        $ (4).\; \mathbf{\mathcal{G}}(\textcolor{blue}{\mathbf{letcont}\; k\; x=t_1\; \mathbf{in}\; t_2}, pc, n, loc) \coloneqq  $ \\
        $ \quad\quad\quad  \mathbf{\mathcal{G}}(t_1, \mathbf{app}_b\; t'\; pc\; k, (pc.l_f, k, 0), n', loc') $ \\
        $ \quad\quad \mathtt{where}\; (t', n', loc') = \mathbf{\mathcal{G}}(t_2, t_{ssa}, pc, n, loc\; (k\mapsto\; t_{cps})) $ \\
        $ (5).\; \mathbf{\mathcal{G}}(\textcolor{blue}{\mathbf{ifz}\; x\; t_1\; t_2}, pc, n, loc) \coloneqq \mathbf{\mathcal{G}}(t_2, \mathbf{app}_b\; t'\; pc\; f_n, (pc.l_f, f_n, 0), n', loc') $ \\
        $ \quad\quad \mathtt{where}\; t_{br} = \mathbf{app}_i\; t_{ssa}\; pc\; [\textcolor{purple}{b_n=\mathbf{icmp}\; x\; 0;\; \mathbf{br_c}\; b_n\; t_n\; f_n;}]  $ \\
        $ \quad\quad \mathtt{and}\; (t', n', loc') = \mathbf{\mathcal{G}}(t_1, \mathbf{app}_b\; t_{br}\; pc\; t_n, (pc.l_f, t_n, 0), n+1, loc) $ \\
        $ (6).\; \mathbf{\mathcal{G}}(\textcolor{blue}{k\; x}, t_{ssa}, pc, n, loc) \coloneqq $\\
        $\quad\quad\quad \left\{ 
        \begin{aligned}
            & (\mathbf{app}_i\; t_{ssa}\; pc\; [\textcolor{purple}{r_b = x;\; \mathbf{ret}\; r_b;}], n, loc) \\[-3pt]
            & \quad\quad \mathtt{when}\; loc\; k \coloneqq \textcolor{blue}{\mathbf{letfun}\; f\; k\; x_0=t_1\; \mathbf{in}\; t_2}\; \mathtt{or}\; k=k_{init} \\[-3pt]
            & (\mathbf{app}_i\; t_{ssa}\; pc\; [\textcolor{purple}{x_0 = x;\; \mathbf{br_{uc}}\; k;}], n, loc) \\[-3pt]
            & \quad\quad \mathtt{when}\; loc\; k \coloneqq \textcolor{blue}{\mathbf{letcont}\; k\; x_0=t_1\; \mathbf{in}\; t_2} \\
        \end{aligned}
        \right. $\\
        % \end{equation}
        $ (7).\; \mathbf{\mathcal{G}}(\textcolor{blue}{f\; k\; x}, t_{ssa}, pc, n, loc) \coloneqq $ \\
        $\quad\quad\quad  \left\{ 
            \begin{aligned}
        & (\mathbf{app}_i\; t_{ssa}\; pc\; [\textcolor{purple}{r_b = \mathbf{call}\; f\; x;\; \mathbf{ret}\; r_b;}], n, loc) \\[-3pt]
        & \quad\quad \mathtt{when}\; loc\; k \coloneqq \textcolor{blue}{\mathbf{letfun}\; f\; k\; x_0=t_1\; \mathbf{in}\; t_2}\; \mathtt{or}\; k=k_{init} \\[-3pt]
        & (\mathbf{app}_i\; t_{ssa}\; pc\; [\textcolor{purple}{x_0 = \mathbf{call}\; f\; x;\; \mathbf{br_{uc}}\; k;}], n, loc) \\[-3pt]      
        & \quad\quad \mathtt{when}\; loc\; k \coloneqq \textcolor{blue}{\mathbf{letcont}\; k\; x_0=t_1\; \mathbf{in}\; t_2} \\
            \end{aligned}
        \right. $\\
    \end{algorithm}
    \caption{CPS到SSA的转换算法}\label{cps2ssa}
\end{figure}

我们设计了将CPS形式的PCF程序转换为SSA程序的转换算法。本质上,该转换过程是一个递归函数,
它接受一个CPS代码项、一个SSA程序及其他信息作为其参数,不断将新的组件(例如基本代码块、指令等)
放入当前SSA程序的特定位置,并在递归地翻译CPS代码项时更新其他参数。
这些参数是确定新的组件具体内容及插入位置所必须的信息。
一旦整个CPS程序的翻译完成,所得到的SSA程序就是我们需要的转换后的目标SSA程序。

在转换算法~\ref{cps2ssa}中,函数$\mathcal{G}_{proc}$将输入的CPS代码项转换为SSA顶层翻译单元$t_{ssa}$。
$t_{ssa}$具有一个主函数,主函数的返回值就是程序的结果。
具体来说,它使用了函数$\mathcal{G}$对源程序进行处理。该函数将$(t_{cps}, t_{ssa}, pc, n, loc)$作为输入,
并输出$(t'_{ssa}, n', loc')$。其中,$t_{cps}$是待转换的CPS程序,$t_{ssa}$是已经生成的SSA程序。
$pc$表示我们插入新的SSA代码片段时,当前程序计数器指向的位置。为了在SSA的条件语句分支中生成新的基本代码块标签,
我们使用参数$n$跟踪CPS程序中已经遇到的$\mathbf{ifz}$代码项的数量。
$loc$指的是从延续变量名到定义它们的CPS代码项的映射。我们可以使用它来确定$k$是局部延续
(即由$\mathbf{letcont}$语句引入的延续变量)还是返回延续(即函数的延续变量)。
请注意,我们在$loc_{cps}$中存储整个$\mathbf{letcont}$或$\mathbf{letfun}$语句来表示局部或返回延续,
尽管我们不需要它们的延续体。这是为了避免引入用于区分局部和返回延续的中间项。

该函数返回更新后的SSA程序$t'_{ssa}$,新的$\mathbf{ifz}$代码项数量$n'$和更新后的映射$loc'$。
在初始状态中,$t_{ssa}$是一个只含有空白主函数的SSA程序,程序计数器$pc$指向主函数的开头,
已处理的条件语句数量$n$为0,映射$loc$为空。
在算法~\ref{cps2ssa}中,$\mathbf{app}$操作表示将新的组件添加到SSA程序的$pc$位置。
其中,$\mathbf{app}_i$表示插入指令,$\mathbf{app}_b$表示插入基本代码块,
$\mathbf{app}_p$表示插入函数。
例如,规则(1)中的$(\mathbf{app}_i\; t_{ssa}\; pc\; [x=v;])$表示在$t_{ssa}$的位置$pc$
处插入指令$[x=v;]$。

我们通过将图中的CPS语言的阶乘程序示例(见图~\ref{factcps})转换为SSA程序(见图~\ref{factssa})
来演示该转换算法的工作原理。
\begin{enumerate}[label={(\alph*)}]
    \item 根据转换规则(3),我们首先将$\mathbf{letfun}$代码项中的$\mathbf{ifz}$语句
        转换为SSA函数$fact$的函数体并添加到初始的空白SSA程序中。
        假设更新后的SSA程序和参数分别是$t_0$、$n_0$和$loc_0$,转换后的下一步就是对
        $\mathcal{G}(\mathbf{letval}\; x_5=2\; \mathbf{in}\dots, t_0, (main,b_1,0), n_0,$ \\ $loc_0)$进行计算。   
    \item 根据转换规则(1),我们在当前SSA程序$t_0$的$(main,b_1,0)$位置处插入赋值指令$[x_5=2;]$,
        并将新的SSA程序称为$t_1$。然后,我们的转换目标就变成了
        $\mathcal{G}(\mathbf{letcont}\; k_2\; y=k_{init}\; y\; \mathbf{in}\; (fact\; k_2\; x_5), t_1, n_0, loc_0)$。
    \item 根据转换规则(4),我们首先应该处理$(fact\; k_2\; x_5)$。
        根据转换规则(7),由于延续$k_2$是一个局部延续,我们将$[y = \mathbf{call}\; fact\; x_5;\; \mathbf{br}_{uc}\; k_2;]$
        这两条指令分别插入到SSA程序的位置$(main,b_1,1)$和$(main,b_1,2)$中。
        然后,我们在SSA程序中添加一个空的基本代码块$k_2$,并将新的SSA程序命名为$t_2$。
        我们还将从$k_2$到$\mathbf{letcont}$代码项的映射添加到$loc_0$中,并将更新后的映射称为$loc_1$。
        这样,转换算法的目标变为$\mathcal{G}(k_{init}\; y, t_2, (main,k_2,0), n_0, loc_1)$。
    \item 根据转换规则(6),我们将指令$[r_{k2} = y;\; \mathbf{ret}\; r_{k2};]$分别插入到
        $(main,k_2,0)$和$(main,k_2,1)$位置中。生成的目标SSA程序如图~\ref{factssa}所示。
\end{enumerate}
